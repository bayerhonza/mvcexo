\documentclass[10pt]{article}
\usepackage{a4wide}   % Mise en page une peu plus grande
\usepackage{hyperref} 
\usepackage{multicol} % pour pouvoir faire du multicols
\usepackage[french]{babel}  %
\usepackage{listings} % Pour faire de jolis listings
\usepackage{pifont}   % Pour les cmds dinglist et dingautolist
\usepackage{graphicx}
\usepackage[utf8]{inputenc} % Pour gérer les accents
\usepackage[T1]{fontenc}      % Gestion de la Cesure des mots accentues

%qsfqsdfq
%%------------------------------------------------------------
%% Défintion de mes commandes
\newcommand{\G}[1]{\og #1 \fg}

%%------------------------------------------------------------
%% Défintion du contenu du titre

\title{\begin{center}
\includegraphics[width=6cm]{ENSIMAG-eps-converted-to.pdf}
\end{center} \vspace{1cm} Compte rendu TP 1}
\author{Auteur : Jan Bayer }

%%------------------------------------------------------------
%% Début du document
%%------------------------------------------------------------
\begin{document}
%% Implantation du titre
\maketitle
%%------------------------------------------------------------
%% Table des matieres
\newpage
\section{Introduction}
Nous avons pour problématique la mise en place d’un programme java permettant la visualisation et la modification d’une base de données à travers des requêtes SQL.
Le programme java doit être sous la forme MVC (modèle, vue, contrôleur).
Dans un premier temps nous allons définir les principes de la programmation orientée objet JAVA, puis on effectuera une analyse sur les différentes démarche que l’on aura suivi et on terminera par décrire les solutions adoptées pour l’implémentation du programme.
\section{Programmation orientée objet et Java}
Java est un langage utilisant l’implémentation objet. L’objet est l’élément sur lequel on voit l’effet de l’action, On peut donc créer des classes et des fonctions pour manipuler les objets.
Une classe est
Une fonction est
ces méthodes peuvent être public c’est à dire visible ou privée.
Elles peuvent abriter des attributs que l’on appel variable. Afin que ces variable soient utilisées elles devront être typées et initialisé.
Un héritage multiple est lorque une nouvelle classe est construite à partir de 2 ou plusieurs ancetres directe

\section{Analyse de projet}
Le but de ce projet est de créer une application qui permettra lire, écrire et modifier les entités et les stocker dans une base de données, tout en utilisant une interface graphique créée en Java. Pour l’implémenter nous utilisons des principes de la programmation orientée objets et des divers motifs de conception. Cette section décrit successivement des concepts utilisés dans notre projets dont MVC, des événements asynchrones et des principes d’interfaces graphiques.

\subsection{MVC}
La majorité des interfaces graphiques est basée sur le motif de Model-View-Controller (MVC) qui était lancé en 1978 et depuis, il devenait un standard pour l'implémentation des interfaces graphiques et des applications web. MVC nous permet de séparer les fonctions et de créer une architecture robuste pour gérer les données ainsi que l’interface graphique d’une manière indépendante. Le MVC est composé de trois éléments fondamentaux:
\begin{itemize}
    \item M - Model - est un composant qui manipule les données et souvent représente un accès au stockage de données (une base de donné, un disque dur …)
    \item V - View - est un composant qui s’occupe des interactions avec un utilisateur et donc contient les éléments graphiques (fenêtre, panel, button …)
    \item C - Controller - est un composant de la logique de l’application.
\end{itemize}

\subsection{Evenements asynchrones}

\section{Implémentation}
\subsection{AWT et Swing}
\subsection{Packaging}
\subsection{ActionListener}


\section{Conclusion}
\end{document}
